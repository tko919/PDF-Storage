\documentclass[dvipdfmx,a4paper]{jsarticle}

%%%%% package %%%%%
\usepackage{amsmath, amssymb,amsthm}
\usepackage{type1cm}
\usepackage{verbatim}
\usepackage{ascmac}
\usepackage{fancybox}
\usepackage{mathrsfs}
\usepackage{physics}
\usepackage[dvipdfmx]{hyperref}
\usepackage{pxjahyper}
\usepackage[all]{xy}
\usepackage[margin=20truemm]{geometry}

%%%%% theorem environment %%%%%
\def\theenumi{\arabic{enumi}}
\def\labelenumi{(\theenumi)}
\theoremstyle{definition}
\newtheorem{thm}{定理}[section]
\newtheorem*{prf}{証明}
% \newtheorem{prob}{問題}[chapter]
% \newtheorem{ans}{解答}[chapter]
\newtheorem{lem}{補題}[section]
\newtheorem{define}{定義}[section]
\newtheorem*{war}{注意}
\newtheorem{prop}{命題}[section]
\newtheorem{cor}{系}[section]
\newtheorem{eg}{例}[section]
\newtheorem{rem}{注意}[section]
\newtheorem{ex}{演習}[section]
\newenvironment{claim}[1]{\par\noindent\underline{Claim:}\space#1}{}

%%%%% command %%%%%
\newcommand{\rad}{{\rm rad}}
\newcommand{\emp}{\emptyset}
\newcommand{\Aut}{{\rm Aut}}
\newcommand{\D}{\mathcal{D}}
\newcommand{\res}{{\rm res}}
\newcommand{\ord}{{\rm ord}}
\newcommand{\rk}{{\rm rank}}
\renewcommand{\deg}{{\rm deg}}
\newcommand{\id}{{\rm id}}
\newcommand{\Fr}{{\rm Frac}}
\newcommand{\p}{\mathfrak{p}}
\newcommand{\Hom}{{\rm Hom}}
\newcommand{\nMods}{{\bf \rm nMods}}
\newcommand{\fMods}{{\bf \rm fMods}}
\newcommand{\sgn}{{\rm sgn}}
\newcommand{\Der}{{\rm Der}}
\newcommand{\Spec}{{\rm Spec}}
\newcommand{\im}{{\rm Im}}
\newcommand{\ctens}{\widehat{\otimes}}
\newcommand{\Ctens}{\widehat{\bigotimes}}
\newcommand{\ch}{{\rm char}}
\renewcommand{\L}{\mathscr{L}}
\newcommand{\der}{\partial}
\renewcommand{\tilde}{\widetilde}
\newcommand{\hlus}{\widehat{\bigoplus}}
\newcommand{\cplus}{\widehat{\oplus}}
\renewcommand{\bar}{\overline}
\newcommand{\ilim}[1][]{\mathop{\varinjlim}\limits_{#1}}
\newcommand{\plim}[1][]{\mathop{\varprojlim}\limits_{#1}}
\newcommand{\Op}{{\rm Op}}
\newcommand{\Ker}{{\rm Ker}}
\newcommand{\pr}{{\rm pr}}
\renewcommand{\id}{{\rm id}}
\newcommand{\Cok}{{\rm Cok}}
\renewcommand{\hat}{\widehat}
\newcommand{\Mor}{{\rm Mor}}
\renewcommand{\ch}{{\rm Char}}
\newcommand{\Ob}{{\rm Ob}}
\newcommand{\resp}{{\rm resp.}}
\newcommand{\GL}{{\rm GL}}
\newcommand{\Vol}{{\rm Vol}}
\newcommand{\dVol}{{\rm dVol}}
\newcommand{\Ric}{{\rm Ric}}
\newcommand{\Rc}{{\rm Rc}}
\newcommand{\Hess}{{\rm Hess}}
\newcommand{\inj}{{\rm inj}}
\renewcommand{\phi}{\varphi}
\renewcommand{\refname}{References}

\newcommand{\N}{\mathbb{N}}
\newcommand{\Z}{\mathbb{Z}}
\newcommand{\Q}{\mathbb{Q}}
\newcommand{\R}{\mathbb{R}}
\newcommand{\C}{\mathbb{C}}
\newcommand{\diag}{{\rm diag}}

%%%%% counter %%%%%
\makeatletter
\let\sectionorig\section
\def\@sectionorig#1{\sectionorig*{\MakeUppercase{#1}}}
\def\@@sectionorig#1{\sectionorig{\MakeUppercase{#1}}}
% \renewcommand{\section}{\@ifstar{\@sectionorig}{\@@sectionorig}}
\makeatother

\begin{document}

\title{Cheegerの有限性定理}
\date{}

\maketitle

\section{目標}
このレポートでは、Cheegerの有限性定理の証明の概略について述べる。

\thm (Cheeger) $n \geq 2,K,D,v>0$ について、「$\mbox{Diam} \leq D,\Vol \geq v,|\sec| \leq K$ なる $n$ 次元閉Riemann多様体の族」は有限個の微分同相類で代表される。

\section{準備}
証明を記すうえで前提となる事実を列挙する。
\begin{enumerate}
    \item Ricci曲率 $\Rc(v,w)=\Tr(x \mapsto R(x,v)w)=\sum_i g(R(e_i,w)e_i)$ ( $\{e_i\}$ :直交基底)
    \item 断面曲率 $\sec(v,w)={g(R(w,v)v,w) \over g(v \wedge w,v \wedge w)}$
    \item 単射半径 $\iota$ : $\exp_p:T_pM \to M$ が微分同相となる半径の上界
\end{enumerate}

\thm (Klingenberg's estimate) $(M,g)$ を完備Riemann多様体とし $\sec \leq C (const.)$ とする。このとき、$\iota_M \geq {\pi \over \sqrt{C}}$ となるか、最短の閉測地線 $\gamma$ が存在して $\iota_M={L(\gamma) \over 2}$ 。

\thm (Ambrose) $(M,g),(N,h)$ をRiemann多様体、$f:M \to N$ を局所等長写像とする。 $(M,g)$ が完備なら $f$ は滑らかな被覆写像である。

\thm (Bishop-Gromov) $(M,g)$ を完備Riemann多様体とし $\Rc \leq (n-1)k$ とする。このとき、$r \mapsto {\Vol B(p,r) \over v(n,k,r)}$ は単調非減少。ここで $v(n,k,r)$ は断面曲率が $k$ で一定の空間上で半径 $r$ の閉球の体積。

\thm\label{thmsplit} (分裂定理) 完備Riemann多様体 $(M,g)$ が直線を含み $\Ric \geq 0$ を満たすとする。このとき、あるRiemann多様体 $(H,g_0)$ が存在して $(M,g)$ は $(H \times \R,g_0+dt^2)$ と等長同型($M$ が\textbf{分裂する}という)。

\section{Riemann多様体の族上のノルム}

基点つきRiemann多様体 $(M,g,p)$ 上の $C^{m,\alpha}$-norm が $||(M,g,p)||_{C^{m,\alpha},r} \leq Q$ であることを、次の条件を満たす $C^{m+1,\alpha}$ -class chart $\phi:B(0,r) \to U \ni p$ が存在することと定義する。 
\begin{enumerate}
    \item $|D\phi|,|D\phi^{-1}| \leq \exp(Q)$
    \item 任意の多重指数 $|I| \leq m$ について $r^{|I|+\alpha}||\der^{I}g_{kl}||_{\alpha} \leq Q$
\end{enumerate}

上のnormを入れた位相を pointed-$C^{m,\alpha}$ 位相と呼ぶ。Riemann多様体の族自体には各点のsupを取ったものがそのままノルムになり、単に $C^{m,\alpha}$ 位相と呼ぶ。

Ascoli-Arzelaの定理に対応する結果が次である。

\thm\label{cpt} $Q>0,n \geq 2,m \geq 0,\alpha \in (0,1],r>0$ とする。「基点つき $n$ 次元Riemann多様体 $(M,g,p)$ で $||(M,g,p)||_{C^{m,\alpha},r} \leq Q$ なるものの族」は pointed-$C^{1,\beta} (\beta<\alpha)$ 位相でコンパクト。 \qed

\cor 上の族の部分集合として、$\mbox{diam} \leq D$ という制限をつけたものは $C^{m,\beta}$ 位相でコンパクトであり、高々有限個の微分同相類を含む。
\prf 直径とノルムの条件から定数個のchartで被覆でき、上の定理と合わせて前半が従う。\\
また、座標変換が $C^1$ ノルムで十分近い二つのRiemann多様体は微分同相であること \cite[Theorem 2.1.6]{hirsch} を用いると、任意の $C^{m,\beta}$ 位相でのコンパクト性は微分同相類の有限性を意味する。\qed

\eg $(M,g)$ を完備平坦なRiemann多様体とすると $\forall r \in \iota(M,g),||(M,g)||_{C^{m,\alpha},r}=0$ 。特に $\forall r>0, ||(\R^n,g_{\R^n})||_{C^{m,\alpha},r}=0$ だが、実は任意の $m,\alpha,r$ で等式が成り立つこととEuclid空間と等長同型であることが後で示される。

\section{調和座標と調和ノルム}
Riemann多様体上の調和座標 $(U,\{x_i\})$ とは、$\Delta x_i=0$ が成り立つことを指す。

\prop Riemann多様体 $(M,g)$ 上の各点 $p$ について調和座標系 $p \in (U,\{x_i\})$ が存在する。

\prf まず適当なchart $(U,\{y_i\})$ を取り $y(p)=0$ とすると、座標変換 $y \mapsto x$ が満たすべき条件は
\begin{align*}
    \Delta x_i={1 \over \sqrt{\det g_{ij}}}\der_i(\sqrt{\det g_{ij}}g^{ij}\der_jx_k)
\end{align*}

この解を見つけるためには、Dirchlet問題 $\Delta x_k=0,x_k=y_k (\mbox{on} \der B(0,\epsilon))$ を解けばよい。 \\
$\{x_k\}$ が実際に座標系となることはelliptic estimatesから従う。\qed

\lem\label{harm}  Riemann多様体 $(M,g)$ 上の調和座標系 $(U,\{x_i\})$ について、次が成り立つ。
\begin{enumerate}
    \item $\Delta u={1 \over \sqrt{\det g_{st}}} \der_i(\sqrt{\det g_{st}} g^{ij}\der_j u)=g^{ij}\der_i\der_j u$
    \item ${1 \over 2}\Delta g_{ij}+Q(g,\der g)=-\Rc_{ij}$ ( $Q$ : 分子が $g$ の多項式と $\der g$ の二次式で分母が $\sqrt{\det g_{ij}}$ にのみ依存する有理多項式)
\end{enumerate}

\prf 
(1) 定義から
\begin{align*}
    0 &= \Delta x^k \\
    &={1 \over \sqrt{\det g_{st}}} \der_i(\sqrt{\det g_{st}} g^{ij}\der_j x^k) \\
    &=g^{ij}\der_i\der_j x^k+{1 \over \sqrt{\det g_{st}}} \der_i(\sqrt{\det g_{st}} g^{ij})\der_j x^k \\
    &=g^{ij}\der_i\delta_j^k+{1 \over \sqrt{\det g_{st}}} \der_i(\sqrt{\det g_{st}} g^{ij})\delta_j^k\\
    &={1 \over \sqrt{\det g_{st}}} \der_i(\sqrt{\det g_{st}} g^{ik}) \\
    \Delta u&={1 \over \sqrt{\det g_{st}}} \der_i(\sqrt{\det g_{st}} g^{ij}\der_j u) \\
    &=g^{ij}\der_i\der_j u+{1 \over \sqrt{\det g_{st}}} \der_i(\sqrt{\det g_{st}} g^{ij})\der_j u \\
    &=g^{ij}\der_i\der_j u
\end{align*}

% TODO
(2) Bochner's formulaを行列表示して得られる。詳細略。
\qed

上の等式をEinstein計量の場合に適用する。つまり $\Rc_{ij}=(n-1)kg_{ij}$ のとき、 ${1 \over 2}\Delta g_{ij}=-(n-1)kg_{ij}-Q(g,\der g)$ 。このとき右辺は $C^1$ 級で意味を持つ式になっている。つまり、$g$ が $C^{1,\alpha}$ 級のとき左辺が $C^{\alpha}$ 級となるが、elliptic estimateから $g$ は $C^{2,\alpha}$ 級になる。この議論を反復することで $g$ のsmoothnessが従う。\\

harmonic norm $||(M,g,p)||^{har}_{C^{m,\alpha},r}$ を、選択するchartに調和性を課したときのノルムと定義する。

次の命題も\ref{cpt}とほとんど同様に示される。収束先でノルムの不等式が保たれることを示す際にDirichlet問題を解く必要がある。

\prop  $Q>0,n \geq 2,m \geq 0,\alpha \in (0,1],r>0$ とする。「基点つき $n$ 次元Riemann多様体 $(M,g,p)$ で $||(M,g,p)||^{har}_{C^{m,\alpha},r} \leq Q$ なるものの族」は pointed-$C^{1,\beta} (\beta<\alpha)$ 位相でコンパクト。\qed

調和座標を用いるメリットとして、計量がRicci曲率によって制御されることが挙げられる。これは次の補題に集約される。

\lem Riemann多様体 $(M,g)$ が有界なRicci曲率 $|\Rc| \leq \Lambda$ を持ち、$\forall r'>r,||(M,g,p)||^{har}_{C^1,r'} \leq K$ が成り立つとき、$\forall \alpha \in (0,1)$ について $C^{1,\alpha}$ ノルムが有界、つまり  $||(M,g,p)||^{har}_{C^{1,\alpha},r} \leq C$ 。

\prf 調和座標を固定し計量成分 $g_{ij}$ を評価する。$\Delta=g^{ij}\der_i \der_j$ に注意する。elliptic estimateから
\begin{align*}
    ||g_{ij}||_{C^{1,\alpha},B(0,r)} \leq C(||\Delta g_{ij}||_{C^0,B(0,r')}+||g_{ij}||_{C^\alpha,B(0,r')})
\end{align*}

\ref{harm} (2)から、
\begin{align*}
    ||\Delta g_{ij}||_{C^0,B(0,r')} \leq 2\Lambda||g_{ij}||_{C^0,B(0,r')}+C'||g_{ij}||_{C^1,B(0,r')}
\end{align*}

上の二つの評価を合わせて結論を得る。 \qed

実は、分裂定理を用いるとharmonic normの評価を単射半径の評価に置き換えることができる。

\thm (Anderson) $n>2,\alpha \in (0,1),\Lambda ,R>0$ が与えられるとき、任意の $Q>0$ について $r>0$ が存在し、「 $|\Rc| \leq \Lambda,\iota \geq R$ なる $n$ 次元閉Riemann多様体の族」は $||(M,g)||^{har}_{C^{1,\alpha},r} \leq Q$ を満たす。

\prf 背理法で示す。ある $Q>0$ と $\forall i \geq 1,(M_i,g_i)$ が存在して、
\begin{align*}
    |\Rc| &\leq \Lambda \\
    \iota &\geq R \\
    ||(M_i,g_i)||^{har}_{C^{1,\alpha},i^{-1}} &> Q
\end{align*}
が成り立つとする。
scaleの連続性からある $r_i \in (0,i^{-1})$ で $||(M,g,p)||^{har}_{C^{m,\alpha},r_i} = Q$ となるので、$\bar{g_i}=r_i^{-2}g_i$ とrescaleすると上の条件は
\begin{align*}
    |\Rc| &\leq r_i\Lambda \\
    \iota &\geq r_i^{-1}R \\
    ||(M_i,g_i)||^{har}_{C^{1,\alpha},1} &= Q
\end{align*}
となる。定義から $||(M_i,g_i,p_i)||^{har}_{C^{1,\alpha},1} \in [{Q \over 2},Q]$ なる $p_i \in M_i$ が存在する。

前の補題から $\forall \gamma \in (0,1)$ について $C^{1,\gamma}$ ノルムで有界なので、\ref{cpt}から pointed- $C^{1,\alpha}$ 位相で収束部分列が取れる。極限を $(M,g,p)$ とすると、ノルムの連続性から $||(M,g,p)||^{har}_{C^{1,\alpha},1} \in [{Q \over 2},Q]$ が成り立つ。

\begin{claim}
    $(M,g)=(\R^n,g_{std})$
\end{claim}

この主張が成り立てば $||(M,g,p)||^{har}_{C^{1,\alpha},1} \in [{Q \over 2},Q]$ と矛盾し背理法が成立する。

収束する調和座標を取ると\ref{harm}から各 $(M_i,\bar{g}_i)$ で ${1 \over 2}\Delta \bar{g}_{kl}+Q(\bar{g},\der \bar{g})=-\Rc_{kl}$ が成り立っているが、Ricci曲率の評価から $|-\Rc| \leq r_i^{-2}\Lambda \bar{g}\to 0$ であり、極限では ${1 \over 2}\Delta g_{kl}+Q(g,\der g)=0$ が分かる。これは $(M,g)$ がEinstein方程式 $\Rc=0$ の弱解であることを意味し、\ref{harm}の系からsmoothでRicci平坦な多様体であることが従う。また、$\iota \geq r_i^{-1}R \to \infty$ であり任意の $(M,g)$ 上の測地線は $(M_i,g_i)$ 上の測地線の極限であることから $\iota(M,g)=\infty$ が分かる。よって分裂定理から $(M,g)$ が標準的なEuclid空間になることが分かった。

\qed

上のRicci曲率に対する評価を断面曲率に対する評価に取り替えて次を得る。
\thm $n\geq 2,\alpha \in (0,1),R,K>0$ が与えられるとき、任意の $Q>0$ について $r>0$ が存在し、「 $|\sec| \leq K,\iota \geq R$ なる $n$ 次元閉Riemann多様体の族」は $||(M,g)||^{har}_{C^{1,\alpha},r} \leq Q$ を満たす。

\section{主定理の証明}

前節の内容から、後は単射半径が制御できれば有限性定理を示すことが出来る。これは半径1の球の体積による評価から実現される。

\lem $n\geq 2,v,K>0$ が与えられるとき、$R>0$ が存在し、「 $|\sec| \leq K,\Vol B(p,1) \geq v (p \in M)$ なる $n$ 次元閉Riemann多様体 $(M,g)$ は $\iota_M \geq R$ を満たす。

\prf 補題の条件を満たし $\iota {M_i} \to 0$ となる列 $(M_i,g_i)$ を取る。$\iota (M_i,p_i)=\iota_{M_i}$ とし $\bar{g_i}=(\iota M_i)^{-2}g_i$ と正規化すると、$\iota(M,\bar{g_i})=1,|\sec(M,\bar{g_i})| \leq (\iota M_i)^{2}K=K_i \to 0$ となる。上の定理からこれは pointed-$C^{1,\alpha}$ 位相で収束部分列を持ち、収束先 $(M,g,p)$ は平坦となる。\\
まず $\iota(M,p) \leq 1$ を導く。 ${\pi \over \sqrt{K_i}} \to \infty$ とKlingenberg's estimateから、 $(M_i,g_i,p_i)$ は長さ2の閉測地線を持つ。これは pointed-$C^{1,\alpha}$ 位相で長さ2の閉測地線に収束するから $\iota \leq 1$ が得られる。 \\
一方で、$\Vol B(p_i,1) \geq v$ とBishop-Gromov不等式から、ある $v'$ が存在し $r<1$ について $\Vol B(p_i,r) \geq v' r^n$ が成り立つ。このとき収束先では $\forall r,\Vol B(p_i,r) \geq v' r^n$ が成り立ち、 $(M,g)$ の平坦性から $(M,g)=(\R^n,g_{std})$ である。 \\
上の主張をもう少し詳しく説明する。Ambroseの定理から普遍被覆 $p:\tilde{M}=\R^n \to M$ が存在し、基本群の元は $\R^n$ の等長写像として実現される。位数が有限の元が存在するとき、その等長写像はある点を中心に回転するものであり、特に固定点を持つ。これは作用が自由であることに反するので基本群はtorsion-free。よって $M$ 自身が単連結でないとするとGalois対応から部分群 $\Z$ に対応する被覆 $\hat{M} \to M$ が取れる。これは $\hat{M}=\R^{n-1} \times S^1$ を意味し、体積の増大度が $r^{n-1}$ のオーダーであることから矛盾。よって $M=\R^n$ であり、 $\iota(M,p) \leq 1$ と合わせて列 $(M_i,g_i)$ の存在が否定される。 \qed



\begin{thebibliography}{9}
    \bibitem{petersen} Petersen, Peter. Riemannian geometry. Vol. 171. New York: Springer, 2006.
    \bibitem{hirsch}Hirsch, Morris W. Differential topology. Vol. 33. Springer Science \& Business Media, 2012.
\end{thebibliography}



\end{document}