\documentclass[dvipdfmx,a4paper]{jsreport}

%%%%% package %%%%%
\usepackage{amsmath, amssymb,amsthm}
\usepackage{type1cm}
\usepackage{verbatim}
\usepackage{ascmac}
\usepackage{fancybox}
\usepackage{mathrsfs}
\usepackage{physics}
\usepackage[dvipdfmx]{hyperref}
\usepackage{pxjahyper}
\usepackage[all]{xy}

%%%%% theorem environment %%%%%
\def\theenumi{\arabic{enumi}}
\def\labelenumi{(\theenumi)}
\theoremstyle{definition}
\newtheorem{thm}{定理}[section]
\newtheorem*{prf}{証明}
\newtheorem{prob}{問題}[chapter]
\newtheorem{ans}{解答}[chapter]
\newtheorem{lem}{補題}[section]
\newtheorem{define}{定義}[section]
\newtheorem*{war}{注意}
\newtheorem{prop}{命題}[section]
\newtheorem{cor}{系}[section]
\newtheorem{eg}{例}[section]
\newtheorem{rem}{注意}[section]
\newtheorem{ex}{演習}[section]

%%%%% command %%%%%
\newcommand{\rad}{{\rm rad}}
\newcommand{\emp}{\emptyset}
\newcommand{\Aut}{{\rm Aut}}
\newcommand{\D}{\mathcal{D}}
\newcommand{\res}{{\rm res}}
\newcommand{\ord}{{\rm ord}}
\newcommand{\rk}{{\rm rank}}
\renewcommand{\deg}{{\rm deg}}
\newcommand{\id}{{\rm id}}
\newcommand{\Fr}{{\rm Frac}}
\newcommand{\p}{\mathfrak{p}}
\newcommand{\Hom}{{\rm Hom}}
\newcommand{\nMods}{{\bf \rm nMods}}
\newcommand{\fMods}{{\bf \rm fMods}}
\newcommand{\sgn}{{\rm sgn}}
\newcommand{\Der}{{\rm Der}}
\newcommand{\Spec}{{\rm Spec}}
\newcommand{\im}{{\rm Im}}
\newcommand{\ctens}{\widehat{\otimes}}
\newcommand{\Ctens}{\widehat{\bigotimes}}
\newcommand{\ch}{{\rm char}}
\renewcommand{\L}{\mathscr{L}}
\newcommand{\der}{\partial}
\renewcommand{\tilde}{\widetilde}
\newcommand{\hlus}{\widehat{\bigoplus}}
\newcommand{\cplus}{\widehat{\oplus}}
\renewcommand{\bar}{\overline}
\newcommand{\ilim}[1][]{\mathop{\varinjlim}\limits_{#1}}
\newcommand{\plim}[1][]{\mathop{\varprojlim}\limits_{#1}}
\newcommand{\Op}{{\rm Op}}
\newcommand{\Ker}{{\rm Ker}}
\newcommand{\pr}{{\rm pr}}
\renewcommand{\id}{{\rm id}}
\newcommand{\Cok}{{\rm Cok}}
\renewcommand{\hat}{\widehat}
\newcommand{\Mor}{{\rm Mor}}
\renewcommand{\ch}{{\rm Char}}
\newcommand{\Ob}{{\rm Ob}}
\newcommand{\resp}{{\rm resp.}}
\newcommand{\GL}{{\rm GL}}
\renewcommand{\phi}{\varphi}

\newcommand{\N}{\mathbb{N}}
\newcommand{\Z}{\mathbb{Z}}
\newcommand{\Q}{\mathbb{Q}}
\newcommand{\R}{\mathbb{R}}
\newcommand{\C}{\mathbb{C}}
\newcommand{\diag}{{\rm diag}}

%%%%% counter %%%%%
\makeatletter
\let\sectionorig\section
\def\@sectionorig#1{\sectionorig*{\MakeUppercase{#1}}}
\def\@@sectionorig#1{\sectionorig{\MakeUppercase{#1}}}
\makeatother

%%%%% title %%%%%
\title{Riemann幾何学とリッチフロー}
\author{tko919}
\date{}


%%%%% document %%%%%
\begin{document}
\maketitle
\tableofcontents

\chapter{Riemann幾何}

\section{Riemann計量と接続}
\define $M$ を $n$ 次元可微分多様体とする。 $M$ 上の\textbf{Riemann計量}とは、各点 $p \in M$ の接ベクトル空間に対する内積 $\langle -,- \rangle_p \colon T_pM \times T_pM \to \R$ の族であり、$\langle -,- \rangle \colon M \ni p \mapsto \langle -,- \rangle_p$ が $C^\infty$ 級であるものを指す。組 $(M,\langle -,- \rangle)$ を\textbf{Riemann多様体}という。


Euclid空間ではベクトルの平行移動が自然に行えるが、一般の多様体上では移動したベクトルが接空間の外にはみ出してしまうことがある。そこで、異なる点の接空間を「つなげる」ことで形式的に平行移動を可能にする機構を作りたい。

\define $M$ を $n$ 次元可微分多様体、$\mathfrak{X}(M)$ を $M$ 上のベクトル場全体とする。

$M$ 上の\textbf{Affine接続}ないし\textbf{共変微分}とは、写像 $\nabla:\mathfrak{X}(M) \times \mathfrak{X}(M) \to \mathfrak{X}(M)$ であって、次の条件を満たすものである。( $X,Y,Z \in \mathfrak{X}(M), f,g \in C^\infty(M,\R)$ とする)

\begin{enumerate}
    \item $\nabla_{fX+gY}Z=f\nabla_XZ+g\nabla_Y Z$
    \item $\nabla_{X}(Y+Z)=\nabla_XY+\nabla_X Z$ 
    \item $\nabla_{X}fY=(Xf)Y+f\nabla_X Y$
\end{enumerate}

この条件を満たす接続は無数に存在する。そこでRiemann計量に対する良い性質を持った接続をひとつ採用したい。

\define Affine接続 $\nabla$ が\textbf{対称的}とは、$X,Y \in \mathfrak{X}(M)$ について $\nabla_XY-\nabla_YX=[X,Y]$ を満たすものをいう。

\define Riemann多様体 $(M,\langle -,- \rangle)$ 上のAffine接続 $\nabla$ が\textbf{計量と両立する}とは、$X,Y,Z \in \mathfrak{X}(M)$ について $X \langle Y,Z \rangle=\langle \nabla_X Y,Z \rangle+\langle Y,\nabla_X Z \rangle$ を満たすものをいう。


接続 $\nabla$ が計量と両立するとき、曲線 $\gamma(t)$ に沿った平行なベクトル場 $X(t),Y(t)$ に対して $\frac{d}{dt}\langle X(t),Y(t) \rangle=\langle \nabla_{d\gamma \over {dt}} X(t),Y(t)\rangle+\langle X(t),\nabla_{d\gamma \over {dt}} Y(t)\rangle=0$ より $\langle X(t),Y(t)\rangle$ は定数になる。つまり、平行移動によって2つのベクトルの「角度」が保存されることを表現している。

\thm Riemann多様体 $(M,\langle -,- \rangle)$ 上で対称的かつ計量と両立するAffine接続 $\nabla$ が一意に存在する。これを\textbf{Levi-Civita接続}という。\qed

\section{測地線と第一変分公式}

\define $(M,\langle -,- \rangle)$ を連結なRiemann多様体、 $\gamma \colon [a,b] \to M$ を区分的なめらかな曲線(つまり、 $[a,b]$ の分割 $\{s_i\}$ が存在して各区間 $[s_i,s_{i+1}]$ 上 $C^\infty$ 級な連続写像)とする。 $\gamma$ の\textbf{長さ} を $L(\gamma) := \sum_{i} \int_{s_i}^{s_{i+1}}|\frac{\der \gamma}{\der s}(s)| ds$ とする($|v|=\sqrt{\langle v,v\rangle}$)。$M$ 上の2点 $p,q$ について $\rho(p,q) := \inf\{L(\gamma) \colon \gamma(a)=p,\gamma(b)=q\}$ が定められ、距離の公理を満たす。また $\gamma$ の\textbf{エネルギー}を $E(\gamma) := \int_{a}^b |\frac{\der \gamma}{\der s}(s)|^2 ds$ とする。$E(p,q)$ も同様に定める。


Cauchy-Schwarzの不等式より $L(\gamma)^2 \leq |b-a|E(\gamma)$ が成り立ち、等号が成立するのは $|\frac{\der \gamma}{\der s}(s)|$ が定数のときである。


端点 $p,q$ を固定したまま $\gamma$ を摂動して、エネルギー $E$ が極小値をとる曲線を調べたい。そのために区分的滑らかな曲線の1パラメータ族 $\hat{\gamma}:[a,b] \times (-\epsilon,\epsilon) \ni (s,u)\mapsto \hat{\gamma}_u(s) \in M$ をとり $\gamma=\hat{\gamma}_0,U(s)={{\der \hat{\gamma}} \over {\der u}}(s,0)$ とおく。逆に $U$ が与えられたとき、条件を満たす $\hat{\gamma}$ を構成することもできる。端点を固定しているので $U(a)=U(b)=0$ に注意する。

ある $U$ について $u \mapsto E(\hat{\gamma})$ の $u=0$ による微分は $\hat{\gamma}$ の取り方に依存しない。よって $\dot{E}(U) := {\der E(\hat{\gamma}) \over \der u}|_{u=0}$ を定義でき、これは $\gamma$ を $U$ に沿って滑らかに変形したときの差分を表す。特に $E(p,q)$ を実現する $\gamma$ は $\dot{E}(U)=0$ を満たすと考えられる。

以降ではベクトル場 $d\hat{\gamma}({\der \over \der u})$ を ${\der \over \der u}$ と略記する。$\nabla$ を Levi-Civita接続とすると、$E$ の被積分関数の $u$ による微分が
\begin{align*}
    {1 \over 2}{\der \over \der u}|{\der \hat{\gamma} \over \der s}|^2 &= \langle \nabla_{\der \over \der u}{\der \over \der s}\hat{\gamma},{\der \hat{\gamma} \over \der s}\rangle \\
    &= \langle \nabla_{\der \over \der s}{\der \over \der u}\hat{\gamma},{\der \hat{\gamma} \over \der s}\rangle \ ( \because [{\der \over \der s},{\der \over \der u}]=0) \\
    &= {\der \over \der s} \langle {\der\hat{\gamma} \over \der u},{\der\hat{\gamma} \over \der s} \rangle-\langle {\der \hat{\gamma} \over \der u},\nabla_{\der \over \der s}{\der \over \der s}\hat{\gamma} \rangle
\end{align*}

となり、これを積分すると
\begin{align}\label{1stvar}
    {1 \over 2}\dot{E}(U) &= \sum_i [\langle U,\dot{\gamma} \rangle]_{s_i}^{s_{i+1}}-\int_{a}^b \langle U,\nabla_{\dot{\gamma}}\dot{\gamma} \rangle ds
\end{align}
が得られる。これを \textbf{第一変分公式}と呼ぶ。

$\hat{\gamma}$ が滑らかなとき第一項が消え、全ての $U$ に対して第一変分が0となるときには \textbf{Euler-Lagrange方程式}
\begin{align}\label{euler}
    \nabla_{\dot{\gamma}}\dot{\gamma}=0
\end{align}
が成り立つ。この方程式を満たす $\gamma$ を\textbf{測地線}という。


改めて $\gamma$ を測地線としたとき、 ${1 \over 2}{d \over ds}|\dot{\gamma}|^2=\langle \nabla_{\der \over \der s}{\der \over \der s} \gamma,\dot{\gamma} \rangle=0$ より $|\dot{\gamma}|$ は定数。よって $s$ は $\gamma$ の弧長パラメータに比例する。また、\eqref{euler}はAffine変換に対して不変なので、 $s\mapsto |\dot{\gamma}|s$ に取り替えることで最初から弧長パラメータ表示を持つと仮定してよい。

\eqref{euler} は2階の常微分方程式なので、初期値 $\gamma(0)=p \in M,\dot{\gamma}(0)=v \in T_pM$ が与えられると、Picard-Lindel\"{o}fの定理より $\exists \delta>0,\exists! \gamma_v:(-\delta,\delta) \to M \mbox{ s.t. }$ \eqref{euler}。この $\delta$ は $(p,v) \in TM$ に対して局所一様連続となるので、結局 $p \in M$ に対して $0 \in \exists U \subset T_pM \mbox{ s.t. }\delta>1$ より \textbf{指数写像} $\exp_p:U \ni v \mapsto \gamma_v(1) \in M$ が定義できる。一般論からこれは $C^\infty$ 級となる。

\lem $\gamma:[0,a]\to M$ を2点 $p,q$ を結ぶ区分的滑らかな曲線で、弧長に比例するパラメータを持つとする。もし $L(\gamma)=\rho(p,q)$ のとき、 $\gamma$ は滑らかで\eqref{euler}を満たす。このとき $\gamma$ を\textbf{最短測地線}と呼ぶ。

\prf パラメータの仮定から $L(\gamma)^2=aE(\gamma)$ より $\gamma$ は $E$ の最小値をとるので $\forall U,\dot{E}(U)=0$ 。$\gamma$ に属する $[0,a]$ の分割 $\{s_i\}$ について、$s_i$ の近傍で0となる適切な $U$ を選ぶことで\eqref{1stvar}の第一項を消すことができ、$\gamma$ のsmoothな点全体で\eqref{euler}が成り立つ。有限集合は測度0なので特に\eqref{1stvar}の第二項は無視できる。よって $\sum_i \langle U(s_i),\dot{\gamma}(s_i^+)-\dot{\gamma}(s_i^-) \rangle=0$ より $\dot{\gamma}(s_i^+)=\dot{\gamma}(s_i^-)$ 。したがって $\gamma$ は $C^1$ 級であり、解の一意性から滑らかな曲線であることがわかる。\qed \\


逆関数定理から $\exp_p$ は $0 \in T_pM$ の近傍 $U$ で局所微分同相となる。$\iota_p:=\inf\{r>0 \colon B(0;r) \subset U\}>0$ を $p$ の\textbf{単射半径}という。また $\iota(V)=\inf_{p \in V} \iota_p$ を $V$ の単射半径と呼ぶが、これは0になることがある。$r \leq \iota_p$ について $\exp_p$ による $B(0;r) \subset T_pM$ は $p$ の局所座標近傍を定める。これを\textbf{正規座標}と呼ぶ。


\lem\label{lemrad} 任意の $p$ について $\iota(V)>0$ なる近傍 $p \in V$ が存在する。

\prf $(p,0) \in TM$ のある近傍 $D$ が存在して $\exp:D \ni (q,v) \mapsto \exp_q(v) \in M \times M$ が定義される。これについて逆関数定理を使うことで所望の結論を得る。 \qed


\lem\label{lemgauss} (Gaussの補題) $p \in M,\delta<\iota_p,v,w \in B(0;\delta) \subset T_pM,q:=\exp_p v \in M$ とすると、微分 $d_v\exp_p:T_v(T_pM) \cong T_pM \to T_qM$ は内積を保存する。つまり、 $\langle d_v\exp_p v,d_v\exp_p w \rangle_q=\langle v,w \rangle_p$ 。

\prf $\epsilon>0$ を十分小さくとり $\phi:[0,1] \times (-\epsilon,\epsilon) \to M,(s,t) \mapsto \exp_p(s(v+tw))$ とおく。このとき
\begin{align*}
    d_v \exp_p(v)&= {\der \over \der s}\exp_p((s+1)v)|_{s=0}={\der \phi \over \der s}(1,0) \\
    d_v \exp_p(w)&= {\der \over \der t}\exp_p(v+tw)|_{t=0}={\der \phi \over \der t}(1,0)
\end{align*}

より示したいことは $\langle {\der \phi \over \der s}(1,0),{\der \phi \over \der t}(1,0) \rangle_q=\langle v,w \rangle_p$ と言い換えられる。

$s \mapsto \phi(s,t)$ は測地線なので $\nabla_{\der \over \der s}{\der \phi \over \der s}(s,t)=0$ 。よって Levi-Civita接続の性質から
\begin{align*}
    {\der \over \der s} \ev{{\der \phi \over \der s},{\der \phi \over \der t}} &= \underset{0}{\underline{\ev{\nabla_{{\der \over \der s}}{\der \phi \over \der s},{\der \phi \over \der t}}}}+\ev{{\der \phi \over \der s},\nabla_{{\der \over \der s}}{\der \phi \over \der t}} \\
    &= \ev{{\der \phi \over \der s},\nabla_{{\der \over \der t}}{\der \phi \over \der s}} \\
    &= {1 \over 2}{\der \over \der t}\ev{{\der \phi \over \der s},{\der \phi \over \der s}} \\
    &= {1 \over 2}{\der \over \der t}|v+tw|^2 = \ev{v,w}+t\ev{w,w}
\end{align*}
より ${\der \over \der s} \ev{{\der \phi \over \der s},{\der \phi \over \der t}}(s,0)=\ev{v,w}$ 。また ${\der \phi \over \der t}(0,0)={\der \over \der t}\exp_p(0)=0$ と併せて $\ev{{\der \phi \over \der s},{\der \phi \over \der t}}(s,0)=s\ev{v,w}$ が分かるので、$s=1$ を代入して題意を得る。 \qed


\prop\label{propmin} $p \in M,r_0 <\iota_p$ とし、 $\gamma:[0,1] \to B(p;r_0):=\exp_p(B(0;r_0)),\gamma(0)=p$ は区分的滑らかな曲線で $q:=\gamma(1)$ とすると、$L(\gamma) \geq r$ 。等号が成り立つのは $\gamma$ が測地線(をパラメータ変換したもの)のときに限る。

\prf $\exists r:[0,1] \to \mathbb{R}_{\geq 0},\eta:[0,1] \to \{X \in T_pM \colon |X|=1\}\mbox{ s.t. }\gamma(t)=\exp_p(r(t)\eta(t))$ より、$\dot{\gamma}(t)=d_{\gamma(t)}\exp_p(\dot{r}(t)\eta(t))+d_{\gamma(t)}\exp_p(r(t)\dot{\eta}(t))$ 。

ここで $\eta$ が微分可能な $t$ では ${d \over dt}\underset{1}{\underline{\ev{\eta(t),\eta(t)}}}=2\ev{\dot{\eta}(t),\eta(t)}=0$ なので補題\ref{lemgauss}より、上の $\dot{\gamma}(t)$ の分解は直交分解。よって

\begin{align*}
    |\dot{\gamma}(t)| &\geq |d_{\gamma(t)}\exp_p(\dot{r}(t)\eta(t))| \\
    &=|\dot{r}(t)| \cdot |d_{\gamma(t)}\exp_p(\eta(t))| =|\dot{r}(t)| \\
    L(\gamma) &\geq \int_{0}^1 |\dot{r}(t)| dt \\
    &\geq |\int_{0}^1 \dot{r}(t) dt| \\
    &= r(1)-r(0)=\rho(p,q)
\end{align*}

等号が成り立つのは $\eta(t)$ が区分的定数かつ $\dot{r}\geq 0$ のときだが、$\eta(t)$ が定数のところでは正規化して $\dot{r}=1$ を仮定してよいので、結局 $\eta$ は定数関数にできる。このとき $\gamma$ は測地線パラメータ変換で得られる。 \qed


\lem\label{lemmid} $r_0<\iota_p$ とし $p,q\in M$ を $\rho(p,q)$ とすると $\exists q' \in \der B(p;r_0)\mbox{ s.t. }\rho(p,q)=\rho(p,q')+r_0$ 。

\prf
$p,q$ を結ぶ曲線の族 $\{\gamma_i\}$ で $L(\gamma_i)\to \rho(p,q)$ なるものをとり、$\der B(p;r_0)$ との交点を $q_i$ とするとコンパクト性から $q_i \to \exists q'$ 。距離の公理からこの $q'$ は条件を満たす。 \qed


\prop\label{propcomp} $(M,g)$ が完備のとき、$v \in T_pM$ を初期値とする\eqref{euler}の解は $[0,\infty)$ 上に延長できる。つまり $\exp_p$ は $T_pM$ 全体で定義される。

\prf 解の存在区間の上限を $A$ としたとき、 $A<\infty$ を仮定して矛盾を言えばよい。$\rho$ の定義より $\rho(\gamma_v(s),\gamma_v(t)) \leq |s-t|\cdot |v|$ 。$s \uparrow A$ とすると $\gamma_v(s)$ はCauchy列なので、完備性から $\exists q=\lim_{s \uparrow A}\gamma_v(s)$ 。ここで補題\ref{lemrad}を使うと測地線が $[0,A+\delta)$ まで延長できるので仮定に矛盾する。 \qed

\thm\label{thmhopf} (Hopf-Rinowの定理) $(M,g)$ が完備のとき、$\forall p,q \in M$ について $p,q$ を結ぶ最短測地線が存在する。

\prf $r_0<\iota_p$ として、 $\rho(p,q)>r_0$ を仮定してよい。補題\ref{lemmid}で得られる $q'$ をとると命題\ref{propmin} から $\exists v \in T_pM,q'=\exp_p(r_0v)$ 。また命題\ref{propcomp}より $\gamma(s)=\exp_p(sv)$ は $[0,\infty)$ 上で定義される。

$K=\{s \in [0,\infty) \colon L(\gamma|_{[0,s]})+\rho(\gamma(s),q)=\rho(p,q)\}$ とおくと、$s_0 \in K$ のとき $L(\gamma|_{[0,s_0]})=\rho(p,q)-\rho(\gamma(s_0),q) \leq \rho(p,\gamma(s_0))$ より $\gamma|_{[0,s_0]}$ は最短測地線。$K$ はコンパクトなので $\bar{s}:= \max K \in K$ であり $q_2=\gamma(\bar{s})$ がとれるので、$q_2=q$ を示せばよい。

そうでないとき、 $r_1=\min(\rho(q_2,q)/2,\iota_{q_2})>0$ とおくと、補題\ref{lemmid}より $\exists q_3 \in \der B(q_2,r_1)\mbox{ s.t. }\rho(q_2,q_3)+\rho(q_3,q)=\rho(q_2,q)$ 。三角不等式より 
\begin{align*}
    \rho(p,q_2)+\rho(q_2,q_3)&=\rho(p,q_2)+\rho(q_2,q)-\rho(q_3,q) \\
    &=\rho(p,q)-\rho(q_3,q) \leq \rho(p,q_3)
\end{align*}

より $\rho(p,q_2)+\rho(q_2,q_3)=\rho(p,q_3)$ 。よって、常微分方程式の解の一意性から $q_2,q_3$ を結ぶ最短測地線は $\gamma$ と一致し、$q_3=\gamma(\bar{s}+r_1)$ より $L(\gamma|_[0,s+r_1])+\rho(q_3,q)=\rho(p,q_2)+\rho(q_2,q_3)+\rho(q_3,q)=\rho(p,q)$ 。これは $\bar{s}$ の最大性に矛盾する。 \qed \\


Riemann多様体に局所的な完備性を仮定しても、同様の結論が得られる。

\define Riemann多様体 $(M,g)$ と1点 $p \in M$ の組 $(M,g,p)$ を \textbf{基点つきRiemann多様体}という。

\define $D>0$ とする。 $\forall l<D$ について基点つきRiemann多様体 $(M,g,p)$ の閉測地球 $\overline{B(p;l)}$ がコンパクトのとき、\textbf{D-完備}という。

\thm D-完備な基点つきRiemann多様体 $(M,g,p)$ について $\exp_p:B(0;D)\to M$ が定義され、$\rho(p,q)<D$ なる $p,q \in M$ を結ぶ最短測地線が存在する。 \qed
 


\section{第二変分公式とJacobi場}
エネルギー汎関数 $E$ が極値をとる曲線を測地線と定義したが、$E$ が極小値をとるならば2階微分 $\ddot{E}(U)={\der^2 E \over \der u^2}|_{u=0}$ は非負となるだろう。

$\gamma:[0,a] \to M$ を測地線とし $\hat{\gamma},U$ をその変分・変分ベクトルとすると、被積分関数の $u$ による2階微分は
\begin{align*}
    {1 \over 2}({\der \over \der u})^2|{\der \hat{\gamma} \over \der s}|^2 &= {\der \over \der u}({\der \over \der s} \langle {\der\hat{\gamma} \over \der u},{\der\hat{\gamma} \over \der s} \rangle-\langle {\der \hat{\gamma} \over \der u},\nabla_{\der \over \der s}{\der \over \der s}\hat{\gamma} \rangle) \\
    &= {\der^2\over \der s \der u}\langle {\der\hat{\gamma} \over \der u},{\der\hat{\gamma} \over \der s} \rangle
    -\underset{0\ (u=0)}{\underline{\langle \nabla_{\der \over \der u}{\der \hat{\gamma} \over \der u},\nabla_{\der \over \der s}{\der \over \der s}\hat{\gamma} \rangle}}
    -\langle {\der \hat{\gamma} \over \der u},\nabla_{\der \over \der u} \nabla_{\der \over \der s}{\der \over \der s}\hat{\gamma} \rangle \\
    &= {\der \over \der s}(\langle \nabla_{\der \over \der u}{\der\hat{\gamma} \over \der u},{\der\hat{\gamma} \over \der s} \rangle
    +\langle {\der\hat{\gamma} \over \der u},\nabla_{\der \over \der u}{\der\hat{\gamma} \over \der s} \rangle)
    -\langle {\der \hat{\gamma} \over \der u},\nabla_{\der \over \der u} \nabla_{\der \over \der s}{\der \over \der s}\hat{\gamma} \rangle
\end{align*}

第三項について曲率テンソルの定義から 
\begin{align*}
    \mathcal{J}_{\gamma}(U) &:=\nabla_{\der \over \der u} \nabla_{\der \over \der s}{\der \over \der s}\hat{\gamma} \\
    &=\nabla_{\der \over \der s} \nabla_{\der \over \der s}{\der \over \der u}\hat{\gamma}+R({\der \over \der u},{\der \over \der s}){\der \over \der s}\hat{\gamma}    
\end{align*}

が成り立ち、これを\textbf{Jacobi作用素}という。これを積分すると\textbf{第二変分公式}
\begin{align*}
    {1 \over 2}\ddot{E}(U) &= [\langle \nabla_{U}U,\dot{\gamma} \rangle
    +\langle U,\nabla_{U}\dot{\gamma} \rangle]_0^a
    -\int_0^a \langle U,\mathcal{J}_{\gamma} U \rangle ds \\
    &= [\langle \nabla_{U}U,\dot{\gamma} \rangle
    ]_0^a
    +\int_0^a (| \nabla_{\dot{\gamma}}U|^2-\langle U,R(U,\dot{\gamma})\dot{\gamma} \rangle) ds
\end{align*}

を得る。

$\mathcal{V}_s$ を $\gamma|_{0,s}$ に沿った区分的滑らかなベクトル場全体の集合、$\mathcal{V}^0_s=\{X\in \mathcal{V}_s \colon X(0)=X(s)=0\}$ とする(特に $\mathcal{V}:=\mathcal{V}_a,\mathcal{V}^0:=\mathcal{V}^0_a$ )。第二変分公式の第二項について、$U,V\in \mathcal{V}$ として
\begin{align*}
    I(U,V) &:=\int_0^a (\langle \nabla_{\dot{\gamma}} U, \nabla_{\dot{\gamma}} V \rangle-\langle R(U,\dot{\gamma})\dot{\gamma} ,V\rangle) ds \\
    &=[\ev{\nabla_{\dot{\gamma}}U,V}]_0^a -\int_0^a \langle \nabla_{\dot{\gamma}}\nabla_{\dot{\gamma}}U+R(U,\dot{\gamma})\dot{\gamma} ,V \rangle ds \\
\end{align*}

を\textbf{指数形式}という。$\forall X \in \mathcal{V}^0,I(X,X) \geq 0$ のとき $I$ を\textbf{半正定値}であるといい、さらに$0\neq \forall X \in \mathcal{V}^0,I(X,X) >0$ のとき $I$ を\textbf{正定値}であるという。$\hat{\gamma}_u(s)=\exp_{\gamma(s)}uX(s)$ を考えることで、$\gamma$ が最短ならば $I$ が半正定値であることが従う。

$\{e_i\}$ を $T_pM$ の正規直交基底として、

\begin{align*}
    \mbox{Rc}(X,Y):=\sum_i \ev{R(X,e_i)e_i,Y}=\tr[Z \mapsto R(Z,X)Y]
\end{align*}

で定まる $TM$ の対称双線形形式を\textbf{Ricci曲率}といい、$R(p)=\sum_i \mbox{Rc}(e_i,e_i)$ で定まる値を\textbf{Scalar曲率}という。

\thm\label{thmmyers} (Myersの定理) $\lambda>0,D>{\pi \over \sqrt{\lambda}}$ とする。D-完備な $n$ 次元Riemann多様体 $(M,g)$ について $\mbox{Rc} \geq (n-1)\lambda$ が成り立てば、 $M$ はコンパクトで $\mbox{diam}(M) \leq {\pi \over \sqrt{\lambda}}$ 

\prf 2点 $p,q \in M$ をとり $\rho(p,q)=l<D$ とする。定理\ref{thmhopf}より最短測地線 $\gamma:[0,1] \to M$ が存在する。$\{e_i\}_{i=1}^n \ (e_1=\dot{\gamma}/l)$ を $\gamma$ に沿った平行なベクトル場で、各点の接空間の正規直交基底を成すとする。そして $U_i=\sin(\pi s)e_i$ とおくと、 $U_i$ は全域で滑らかなことに注意して
\begin{align*}
    I(U_i,U_i) &=-\int_0^1 \ev{U_i,\nabla_{\dot{\gamma}}\nabla_{\dot{\gamma}}U_i+R(U_i,\dot{\gamma})\dot{\gamma}} ds \\
    &=-\int_0^1 {\der^2 \sin(\pi s) \over \der s^2}+l^2\sin^2(\pi s) \ev{R(e_i,\dot{\gamma})\dot{\gamma},e_i} ds \\
    &=\int_0^1 \sin^2(\pi s)(\pi^2-l^2 \ev{R(e_i,\dot{\gamma})\dot{\gamma},e_i}) ds \\
    \sum_{i=2}^n I(U_i,U_i) &=\int_0^1 \sin^2(\pi s)((n-1)\pi^2-l^2 \mbox{Rc}(\dot{\gamma},\dot{\gamma})) ds
\end{align*}

ここで $l>\pi/\sqrt{\lambda}$ とすると右辺は負になるので、$\exists U_i,I(U_i,U_i)<0$ 。よって $I$ は半正定値でないので $\gamma$ の最短性に矛盾する。定理\ref{thmhopf}から $M=\exp_p(\bar{B(0;\pi/\sqrt{\lambda})})$ より $M$ はコンパクト。 \qed

\define $\gamma$ に沿ったベクトル場 $V$ が 
\begin{align}\label{jacobi}
    \mathcal{J}_\gamma V &=\nabla_{\dot{\gamma}}\nabla_{\dot{\gamma}}V+R(V,{\dot{\gamma}}){\dot{\gamma}}=0
\end{align}

を満たすとき、 $V$ を\textbf{Jacobi場}という。
\ref{jacobi}は2階の常微分方程式なので、基点 $p=\gamma(0)$ での初期値 $V(0),\nabla_{\dot{\gamma}}V(0) \in T_pM$ が与えられたとき、条件を満たすJacobi場が一意的にとれる。よって、Jacobi場全体は $2n$ 次元のベクトル空間をなす。

\lem\label{lemnormal} $V$ を $\gamma$ に沿ったJacobi場とする。
$\ev{V,\dot{\gamma}}=0$ と $\ev{V(0),\dot{\gamma}(0)}=\ev{\nabla_{\dot{\gamma}}V(0),\dot{\gamma}(0)}=0$ は同値(このような $V$ を\textbf{直交Jacobi場}という)。

\prf ${d^2 \over dt^2}\ev{V,\dot{\gamma}}=\ev{\nabla_{\dot{\gamma}}\nabla_{\dot{\gamma}}V,\dot{\gamma}}=-\ev{R(\dot{\gamma},V)\dot{\gamma},\dot{\gamma}}=0$ より $\ev{V,\dot{\gamma}}=ct+d\ (c,d \in \R)$ と書ける。 $U=V-{ct+d \over |\dot{\gamma}|^2}\dot{\gamma}$ とおくと $\ev{U,\dot{\gamma}}=0$ より $U$ は直交Jacobi場となる。このとき $\ev{V(0),\dot{\gamma}(0)}=d,\ev{\nabla_{\dot{\gamma}}V(0),\dot{\gamma}(0)}=c$ より題意を得る。\qed


\eg\label{egjacobi} $\gamma(s,u)$ を測地線の1パラメータ族とすると $\nabla_{\der \over \der s} \dot{\gamma}$ 。よってJacobi作用素の定義から $U={\der \gamma \over \der u}|_{u=0}$ 。特に $\gamma(s,t)=\exp_p (s(X+uY))$ のとき $U=d_{sX}\exp_{p}(sY)$ であり、初期値は $U(0)=0,\nabla_{\dot{\gamma}}U(0)=Y$ なので $U(0)=0$ なるJacobi場は全てこの形になる。

\lem\label{lemjacobi} 測地線 $\gamma:[0,a] \to M$ の指数形式が半正定値で $U \in \mathcal{V}^0,I(U,U)=0$ ならば、$U$ はJacobi場。

\prf $\forall V \in \mathcal{V}^0,\forall \alpha \in \mathbb{R}$ について $I(U+\alpha V,U+\alpha V)=2\alpha I(U,V)+\alpha^2I(V,V)\geq 0$ より $I(U,V) =0$ (そうでないなら、$\alpha$ を適切に設定して左辺を負にすることが出来る)。よって、
\begin{align*}
    0=I(U,V)=\sum_i \ev{\nabla_{\dot{\gamma}}U(s_i^-)-\nabla_{\dot{\gamma}}U(s_i^+),V}-\int_0^a \ev{\mathcal{J}_\gamma U,V} ds
\end{align*}
が成り立ち、 $V$ は任意なので $\mathcal{J}_\gamma U=0$ が従う。 \qed

\lem\label{lemindex} (指数定理) 測地線 $\gamma:[0,a] \to M$ の指数形式が半正定値とする。$\gamma$ に沿ったJacobi場 $Y$ と $V \in \mathcal{V}$ が $Y(0)=V(0),Y(a)=V(a)$ を満たすとき、$I(Y,Y) \leq I(V,V)$ が成り立つ。等号が成立する条件は $V$ がJacobi場であるときのみ。

\prf 指数形式の定義より $I(Y,Y-V)=0$ なので
\begin{align*}
    0 &\leq I(Y-V,Y-V) \\
    &=I(-Y-V,Y-V)+2I(Y,Y-V) \\
    &=I(V,V)-I(Y,Y)
\end{align*}
等号条件は前の補題とJacobi場の線形性から成り立つ。 \qed

\define 測地線 $\gamma$ 上の相異なる2点 $p=\gamma(0),q=\gamma(s_1)$ に対して $V(0)=V(s_1)=0$ なる $\gamma$ 上の0でないJacobi場が存在するとき、 $p,q$ は $\gamma$ に沿って \textbf{共役} であるという。例\ref{egjacobi}より $\gamma_X$ に沿って $p$ と $q=\exp_p(X)$ が共役なことと $X$ が $\exp_p$ の臨界点であることは同値。

\prop\label{propconj} $\gamma:[0,a] \to M$ を測地線とする。
\begin{enumerate}
    \item $\gamma$ 上 $p=\gamma(0)$ と共役な点が存在しない $\iff$ $I$ が正定値
    \item $\gamma$ 上 $p=\gamma(0)$ と共役な $q=\gamma(a)$ でない点が存在しない $\iff$ $I$ が半正定値
\end{enumerate}

\prf 必要性を示す。$q_1=\gamma(s_1)$ とし、$V$ を $\gamma$ 上のJacobi場で $V(p)=V(q_1)=0$ を満たすとする。このとき、$U(s)=\begin{cases}
    V(s)&(s \in [0,s_1]) \\
    0 & (\mbox{otherwise})
\end{cases}$ は $\gamma$ 上に区分的滑らかなベクトル場であり $I(U,U)=0$ を得る。$I$ が正定値ならば $V \equiv 0$ である。半正定値ならば補題\ref{lemjacobi}より $U$ はJacobi場だが、$s_1 \neq a$ のとき $U$ は $a$ の近傍で恒等的に0なので $U(a)=\nabla_{\dot{\gamma}}U(a)=0$ 。初期値を満たすJacobi場の一意性より $V \equiv 0$ を得る。

十分性を示す。「$I$ が半正定値でないなら端点でない $p$ と共役な点が存在する」ことを示せばよい。($\because \ $ 特に $\gamma$ 上 $p$ と共役な点が存在しないときも $I$ は半正定値なので、$V \in \mathcal{V}^0,I(V,V)=0$ について補題\ref{lemjacobi}より $V$ はJacobi場。$V(0)=V(a)=0$ と仮定から $V \equiv 0$ が従い $I$ は正定値となる)
$\gamma_s:=\gamma|_{[0,s]}$ に対し $I_s:=I_{\gamma_s}$ が半正定値となる $s$ の上限を $s_1$ とおく。補題\ref{lemrad}、命題\ref{propmin}より $s_1>0$ であり、 $I$ の連続性から $I_{s_1}$ も半正定値。$s_1<a$ ならば $p,\gamma(s_1)$ が共役であることを言いたい。

再び補題\ref{lemrad}より $0<\forall \epsilon<\delta$ について $\sigma_{\epsilon}=\gamma|_{[s_1-\delta,s_1+\epsilon]}$ が最短となる $\delta>0$ が存在する。このとき $I_{s_1+\epsilon}$ は半正定値でないので、 $\exists X_\epsilon \in \mathcal{V}^0_{s_1+\epsilon},I_{s_1+\epsilon}(X_\epsilon,X_\epsilon)<0$ 。$s_1-\delta$ の近傍での変形によって $|X_\epsilon(s_1-\delta)|=1$ を仮定してよい。例\ref{egjacobi}より $U^-(0)=0,U^-(s_1-\delta)=X_\epsilon(s_1-\delta)$ なる $\gamma_{s_1-\delta}$ 上のJacobi場が存在し、補題\ref{lemindex}より $I_{s_1-\delta}(U^-,U^-) \leq I_{s_1-\delta}(X_\epsilon,X_\epsilon)$ 。同様に $U^+(s_1+\epsilon)=0,U^+(s_1-\delta)=X_\epsilon(s_1-\delta)$ なる $\sigma_{\epsilon}$ 上のJacobi場をとると $I_{\sigma_{\epsilon}}(U^+,U^+) \leq I_{\sigma_{\epsilon}}(X_\epsilon,X_\epsilon)$ が分かる。

$U^-,U^+$ を $s_1-\delta$ で接続したベクトル場を $V_\epsilon \in \mathcal{V}^0_{s_1+\epsilon}$ とすると、辺々足して $I_{s_1+\epsilon}(V_\epsilon,V_\epsilon) \leq I_{s_1+\epsilon}(X_\epsilon,X_\epsilon)<0$ 。$\epsilon \downarrow 0$ なる列を適切に選び $U^-,U^+$ を収束させると、 $|V_\epsilon(s_1-\delta)|=1$ より $V_\epsilon$ は0でないベクトル場 $V \in \mathcal{V}^0_{s_1}$ に収束する。これは $I_{s_1}(V,V) \leq 0$ を満たすが、$I_{s_1}$ は半正定値なので0に等しく、補題\ref{lemjacobi}より $V$ はJacobi場。よって $p,\gamma(s_1)$ は共役。
\qed

\lem\label{lemnbhd} D-完備なRiemann多様体 $(M,g)$ 上の2点 $p,q=\exp_p v$ 間の最短測地線 $\gamma:s \mapsto \exp_p sv$ が  $|v|<D$ であり、$p,q$ が $\gamma$ 上共役でないとする。このとき
\begin{enumerate}
    \item $v \in T_pM$ の近傍 $U$ が存在し、$w \in U$ ならば測地線 $s \mapsto \exp_p sw$ は $[0,1]$ 上最短。
    \item $p,q$ を結ぶ最短測地線は $\gamma$ のみ。
\end{enumerate}

\prf $p,q$ は共役でないので $\exp_p$ は $v$ で正則。よって逆関数定理より $v$ の近傍 $U$ と $q$ の近傍 $V$ の微分同相を導く。条件(1) が成立するとき $\epsilon>0$ が存在して $\gamma$ は $[0,1+\epsilon]$ まで延長できるので、
常微分方程式の一意性から(2)がわかる。

逆に (1) が成り立たないとき、$\gamma$ に収束し $[0,1]$ 上最短でない測地線の列 $\{\gamma_i\}$ がとれる。$p,q_i=\gamma_i(1)$ を結ぶ最短測地線を $\sigma_i$ とおくと、$q_i \in V$ を仮定してよいので $w_i:=\dot{\sigma_i}(0) \notin U$ となる必要がある。このとき $w_i \to w \in \bar{B(0;D)} \setminus U$ となるように列を取り直すと、 $\sigma_i \to (s \mapsto \exp_p(sw)) \neq \gamma$ より (2) も不適。 \qed

\define $v \in S_pM$ について $\mbox{cut}(v):=\sup\{s_0 \in [0,D) \colon [0,s_0] \ni s \mapsto \exp_p(sv)\mbox{が最短}\}$  とおく。$q=\exp_p(\mbox{cut}(v)v)$ を $p$ の\textbf{切片} と呼ぶ。 命題\ref{propmin}よりこれは正であり、特に $(M,g)$ が完備なら $\mbox{cut}(v)=\infty$ になり得る。このとき $s \mapsto \exp_p(sv)$ はどの有界区間に制限しても最短になる。このような測地線を\textbf{半直線}と呼ぶ。

\lem $(M,g)$ が完備Riemann多様体のとき、$(M,g)$ がコンパクト $\iff$ $p$ を始点とする半直線が存在しない。

\prf コンパクトならば有界なので十分性は明らか。逆に非コンパクトのとき $\rho(p,q_i) \to \infty$ なる点列を取り、$\gamma_i$ を $p,q_i$ を結ぶ弧長パラメータ表示を持つ最短測地線とする。このとき $(M,g)$ の完備性と $S_pM$ のコンパクト性から $\gamma_i$ が $[0,\infty)$ 上の測地線 $\gamma$ に収束するようにでき、これは半直線となる。 \qed

\cor\label{corsmth} $s_0<\mbox{cut}(v)$ に対して $p,q=\exp_p(s_0v)$ を結ぶ最短測地線は一意で、 $q$ の近傍で $\rho(p,-)$ は滑らか。

\prf 補題\ref{lemnbhd} と常微分方程式の一意性から従う。 \qed

\cor $q$ が $p$ の切片 $\iff$ $p,q$ を結ぶ最短測地線上で $p,q$ が共役、または $p,q$ を結ぶ最短測地線が複数存在する。

\prf 命題 \ref{propconj}、補題\ref{lemnbhd} と常微分方程式の一意性から従う。 \qed

\lem $S_pM \ni v \mapsto \mbox{cut}(v)$ は連続。

\prf $v_i \to v$ とする。最短測地線の収束性から $\limsup_{i \to \infty} \mbox{cut}(v_i) \leq \mbox{cut}(v)$ 。一方、系\ref{corsmth}より $l<\mbox{cut}(v)$ について $\gamma:[0,l] \to M,s \mapsto \exp_p(sv)$ は線分で前の系から端点は共役でない。したがって補題\ref{lemnbhd}より $\gamma_i:s \mapsto \exp_p(sv)$ も $[0,l]$ 上で最短測地線なので $\liminf_{i \to \infty} \mbox{cut}(v_i) \geq \mbox{cut}(v)$ 。 \qed

\section{Rauchの比較定理}

$(M,g),(\tilde{M},\tilde{g})$ を $m$ 次元Riemann多様体、$\gamma:[0,a] \to M,\tilde{\gamma}:[0,a] \to \tilde{M}$ を $\gamma(0)=p,\tilde{\gamma}(0)=\tilde{p}$ なる正規測地線とする。 $t \in [0,a]$ について $K^-(t)=\min\{K(\Pi_{\gamma}) \colon \dot{\gamma}(t) \in \Pi_{\gamma}\},\tilde{K}^+(t)=\min\{\tilde{K}(\tilde{\Pi}_{\tilde{\gamma}}) \colon \dot{\tilde{\gamma}}(t) \in \tilde{\Pi}_{\tilde{\gamma}}\}$ とおく(ここで $K,\tilde{K}$ は断面曲率)。

\lem $X,\tilde{X}$ を$\gamma,\tilde{\gamma}$ に沿った直交Jacobi場とし、
\begin{enumerate}
    \item $X(0)=\tilde{X}(0)=0$ 
    \item $|X(a)|=|\tilde{X}(a)|$
\end{enumerate}

さらに定理の(4),(5)を満たすとき、$I(X,X)\leq I(\tilde{X},\tilde{X})$ が成り立つ。

\prf $\{e_i(t)\},\{\tilde{e_i}(t)\}$ を $\gamma,\tilde{\gamma}$ に沿った正規直交基底とし、$e_1=\dot{\gamma},\tilde{e_1}=\dot{\tilde{\gamma}},e_2(a)=X(a)/|X(a)|,\tilde{e_2}(a)=\tilde{X}(a)/|\tilde{X}(a)|$ を満たすとする。ここで、(4)より $|X(a)|=|\tilde{X}(a)|=\alpha \neq 0$ に注意する。

$X,\tilde{X}$ の $\{e_i(t)\},\{\tilde{e_i}(t)\}$ による成分表示を $\{X^i(t)\},\{\tilde{X}^i(t)\}$ とおくと、条件から
\begin{enumerate}
    \item $\forall i,X^i(0)=\tilde{X}^i(0)=0$
    \item $X^2(a)=\tilde{X}^2(a)=\alpha,\forall i \neq 2,X^i(a)=\tilde{X}^i(a)=0$
    \item $\forall t\in [0,a],X^1(t)=\tilde{X}^1(t)=0$ 
\end{enumerate}

が成り立つ。$Y:=\sum_i \tilde{X}^i(t)e_i(t)$ とおくと $Y(0)=0,Y(a)=X(a)$ であり、$X$ はJacobi場なので補題\ref{lemindex}より $I(X,X) \leq I(Y,Y)$ 。一方、(5) を用いて $I(Y,Y) \leq I(\tilde{X},\tilde{X})$ が分かるので、併せて題意を得る。 \qed 

\thm\label{thmrauch} (Rauchの比較定理) $X,\tilde{X}$ を $\gamma,\tilde{\gamma}$ のJacobi場とし、次の条件を満たすと仮定する。
\begin{enumerate}
    \item $X(0)=\tilde{X}(0)=0$
    \item $|\nabla_{\dot{\gamma}} X(0)|=|\tilde{\nabla}_{\dot{\tilde{\gamma}}} \tilde{X}(0)|$
    \item $\ev{\dot{\gamma}(0),\nabla_{\dot{\gamma}}X(0)}=\ev{\dot{\tilde{\gamma}}(0),\tilde{\nabla}_{\dot{\tilde{\gamma}}} \tilde{X}(0)}$
    \item $\gamma$ には $[0,a]$ 上共役な点対が存在しない
    \item $\forall t \in [0,a],\tilde{K}^+(t) \leq K^-(t)$ 
\end{enumerate}

このとき、$\tilde{\gamma}$ には $[0,a]$ 上共役な点対が存在せず、$\forall t\in [0,a],|X(t)|\leq |\tilde{X}(t)|$ 。

\prf (1),(3)より初期値の接線成分が一致することから、一般性を失わず $X,\tilde{X}$ は直交Jacobi場としてよい(補題\ref{lemnormal}を参照)。

$u(t)=|X(t)|^2,\tilde{u}(t)=|\tilde{X}(t)|^2$ とおくと、$\tilde{u}(t)/u(t)$ は $(0,a)$ 上well-definedであり、L'Hospitalの定理より
\begin{align*}
    \lim_{t \to 0}{{\tilde{u}(t)} \over {u}(t)} &= \lim_{t \to 0} {\ddot{\tilde{u}(t)} \over \ddot{u}(t)} \\
    &= {|\nabla_{\dot{\gamma}}X(0)|^2 \over |\nabla_{\dot{\tilde{\gamma}}}\tilde{X}(0)|^2}=1
\end{align*}
よって $|X| \leq |\tilde{X}|$ を示すためには ${d \over dt}{\tilde{u}(t) \over u(t)} \geq 0 \iff \dot{\tilde{u}}(t)u(t)-\tilde{u}(t)\dot{u}(t) \geq 0$ が分かればよい。

(4)より $(0,a]$ 上 $u(t)>0$ が成り立つ。 $c=\sup\{t \colon \tilde{u}(t)>0\} \leq a$ とおき、 $b \in (0,c)$ について $X_b(t)={X(t) \over |X(b)|},\tilde{X}_b(t)={\tilde{X}(t) \over |\tilde{X}(b)|}$ とすると、$[0,b]$ 上 $X_b,\tilde{X}_b$ は前補題の条件を満たすので $I(X_b,X_b) \leq I(\tilde{X}_b,\tilde{X}_b)$ 。指数形式の定義(に部分積分を施した形)から $I(X_b,X_b)=\ev{\nabla_{\dot{\gamma}}X_b(b),X_b(b)} \leq I(\tilde{X}_b,\tilde{X}_b)=\ev{\tilde{\nabla}_{\dot{\tilde{\gamma}}}\tilde{X}_b(b),\tilde{X}_b(b)}$ となるので、$\forall b \in (0,c)$ で
\begin{align*}
    {1 \over 2}{\dot{u}(b)\over u(b)} &= {\ev{\nabla_{\dot{\gamma}}X_b(b),X_b(b)} \over \ev{X(b),X(b)}} \\
    &\leq {\ev{\tilde{\nabla}_{\dot{\tilde{\gamma}}}\tilde{X}_b(b),\tilde{X}_b(b)} \over \ev{\tilde{X}(b),\tilde{X}(b)}} \\
    &= {1 \over 2}{\dot{\tilde{u}}(b)\over \tilde{u}(b)}
\end{align*}

が成り立つ。これを移項して欲しかった不等式を得る。

もし $c<a$ ならば連続性から $|\tilde{X}(c)| \geq |X(c)|>0$ となり $c$ の選び方に矛盾。したがって $c=a$ より $\tilde{\gamma}$ は $[0,a]$ 上共役な点対が存在しない。 \qed
\\

この定理は定曲率を持つ単連結Riemann多様体をモデルとし、考察しているRiemann多様体 $(M,g)$ の曲率と大小関係を持つという仮定のもと調べる、という形で応用されることが多い。










\end{document}