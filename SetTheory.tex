\documentclass[dvipdfmx,a4paper]{jsreport}

%%%%% package %%%%%
\usepackage{amsmath, amssymb,amsthm}
\usepackage{type1cm}
\usepackage{verbatim}
\usepackage{ascmac}
\usepackage{fancybox}
\usepackage{mathrsfs}
\usepackage[all]{xy}

%%%%% theorem environment %%%%%
\def\theenumi{\arabic{enumi}}
\def\labelenumi{(\theenumi)}
\theoremstyle{definition}
\newtheorem{thm}{定理}[section]
\newtheorem*{prf}{証明}
\newtheorem{prob}{問題}[chapter]
\newtheorem{ans}{解答}[chapter]
\newtheorem{lem}{補題}[section]
\newtheorem{define}{定義}[section]
\newtheorem*{war}{注意}
\newtheorem{prop}{命題}[section]
\newtheorem{cor}{系}[section]
\newtheorem{eg}{例}[section]
\newtheorem{rem}{注意}[section]

\newtheorem{ex}{演習}[section]

%%%%% command %%%%%
\newcommand{\rad}{{\rm rad}}
\newcommand{\emp}{\emptyset}
\newcommand{\Aut}{{\rm Aut}}
\newcommand{\D}{\mathcal{D}}
\newcommand{\res}{{\rm res}}
\newcommand{\ord}{{\rm ord}}
\newcommand{\rk}{{\rm rank}}
\renewcommand{\deg}{{\rm deg}}
\newcommand{\id}{{\rm id}}
\newcommand{\Fr}{{\rm Frac}}
\newcommand{\p}{\mathfrak{p}}
\newcommand{\Hom}{{\rm Hom}}
\newcommand{\nMods}{\textbf{\rm nMods}}
\newcommand{\fMods}{\textbf{\rm fMods}}
\newcommand{\sgn}{{\rm sgn}}
\newcommand{\Der}{{\rm Der}}
\newcommand{\Spec}{{\rm Spec}}
\newcommand{\im}{{\rm Im}}
\newcommand{\ctens}{\widehat{\otimes}}
\newcommand{\Ctens}{\widehat{\bigotimes}}
\newcommand{\ch}{{\rm char}}
\renewcommand{\L}{\mathscr{L}}
\newcommand{\der}{\partial}
\renewcommand{\tilde}{\widetilde}
\newcommand{\hlus}{\widehat{\bigoplus}}
\newcommand{\cplus}{\widehat{\oplus}}
\renewcommand{\bar}{\overline}
\newcommand{\ilim}[1][]{\mathop{\varinjlim}\limits_{#1}}
\newcommand{\plim}[1][]{\mathop{\varprojlim}\limits_{#1}}
\newcommand{\Op}{{\rm Op}}
\newcommand{\op}{{\rm op}}
\newcommand{\Ker}{{\rm Ker}}
\newcommand{\pr}{{\rm pr}}
\renewcommand{\id}{{\rm id}}
\newcommand{\Cok}{{\rm Cok}}
\renewcommand{\hat}{\widehat}
\newcommand{\Mor}{{\rm Mor}}
\renewcommand{\ch}{{\rm Char}}
\newcommand{\Ob}{{\rm Ob}}
\newcommand{\resp}{{\rm resp.}}
\newcommand{\GL}{{\rm GL}}
\renewcommand{\phi}{\varphi}

\newcommand{\N}{\mathbb{N}}
\newcommand{\Z}{\mathbb{Z}}
\newcommand{\Q}{\mathbb{Q}}
\newcommand{\R}{\mathbb{R}}
\newcommand{\C}{\mathbb{C}}
\newcommand{\diag}{{\rm diag}}

%%%%% counter %%%%%
\makeatletter
\let\sectionorig\section
\def\@sectionorig#1{\sectionorig*{\MakeUppercase{#1}}}
\def\@@sectionorig#1{\sectionorig{\MakeUppercase{#1}}}
\renewcommand{\section}{\@ifstar{\@sectionorig}{\@@sectionorig}}
\makeatother

%%%%% title %%%%%
\title{集合論 メモ}
\author{tko919}
\date{}


%%%%% document %%%%%
\begin{document}
\maketitle
\tableofcontents

\chapter{順序数}

\section{ZFC公理系}
\begin{enumerate}
    \item 外延性の公理 $\colon$ 集合 $A,B$ に属する元が一致するなら $A=B$
    \item 空集合の公理 $\colon$ いかなる元も持たないような集合が存在する
    \item 対の公理 $\colon$ 要素 $x,y$ について、$x,y$ のみを元とする集合が存在する
    \item 和集合の公理 $\colon$ 集合 $X$ について、 $X$ の元の要素全体からなる集合が存在する
    \item 無限公理 $\colon$ $\emptyset \in X$ かつ $\forall x \in X,x \cup \{x\} \in X$ なる集合 $X$ が存在する
    \item べき集合の公理 $\colon$ 集合 $X$ について、 $X$ の部分集合全体からなる集合 $2^X$ が存在する
    \item 置換公理 $\colon$ 集合 $X$ と論理式 $\phi$ について、 $\forall x \in X$ に対し $\phi(x,y)$ を満たす $y$ が一意に存在するなら、$\{y | \exists x \in X,\phi(x,y)\}$ は集合である
    \item 正則性公理 $\colon$ 空でない集合 $X$ について、元 $x\in X$ であって $\forall y \in X,y \notin x$ が成り立つ
    \item 選択公理 $\colon$ 空でない集合の族 $\{X_\lambda\}_{\lambda \in \Lambda}$ について、選択関数 $f:\Lambda \to \cup_{\lambda \in \Lambda}X_{\lambda},f(\lambda) \in X_{\lambda}$ が存在する
\end{enumerate}

\section{整列集合の諸性質}
\define 全順序かつ任意の部分集合が最小元を持つとき、\textbf{整列集合} と呼ぶ。

\prop\label{monotone} $(P,<)$ を整列集合, $f:P \to P$ を単調増加とすると, $\forall x \in P,f(x) \geq x$ 。
\prf $f(x)<x$ なる $x \in P$ の最小元を $z$ とする。 $w=f(z)$ とおくと $f(w)<f(z)=w<z$ より最小性に矛盾。 \qed

\cor 整列集合の同型写像は恒等写像しかない。 \qed
\cor 整列集合 $P_1,P_2$ が同型ならば、同型写像は一意。 \qed \\
\\
$P$ を整列集合,$w \in P$ として, $\{x \in P \colon x<w \}$ を $w$ の \textbf{切片} $P(w)$ と呼ぶ。

\lem\label{segment} 自身の切片と同型な整列集合 $P$ は存在しない。
\prf $f:P \to P(w)$ を同型写像とすると, $f(w)<w,P(w) \subset P$ より命題 \ref{monotone} に矛盾する。 \qed

\thm $P_1,P_2$ :整列集合について, 次のいずれか1つが成り立つ。
\begin{itemize}
    \item $P_1,P_2$ は同型
    \item $P_1$ と $P_2$ の切片は同型
    \item $P_2$ と $P_1$ の切片は同型
\end{itemize}
\prf $f=\{(x,y) \in P_1 \times P_2 \colon P_1(x) \cong P_2(y)\}$ とおく。 \\
補題 \ref{segment} より $f$ は単射な写像。また $x'<x$ とすると、同型写像 $\phi:P_1(x) \to P_2(f(x))$ を $P_1(x')$ に制限して $P_1(x') \cong P_2(\phi(x')),\phi(x')=f(x')<f(x)$ を得るので, $f$ は順序を保つ。 \\
$\mbox{dom} f=P_1,\mbox{ran} f=P_2$ のとき1番目が成り立つ。 \\
$\mbox{ran}f \neq P_2$ のとき、$y_0$ を $P_2 \setminus \mbox{ran}f$ の最小元とする。 $y_1<y_2,y_2 \in \mbox{ran}f \Rightarrow y_1 \in \mbox{ran} f$ に注意すると $\mbox{ran}f \cong P_2(y_0)$ がわかる。 もし $\mbox{dom}f \neq P_1$ ならば $x_0$ を $P_1 \setminus \mbox{dom} f$ の最小元として同様に $\mbox{dom}f \cong P_1(x_0)$ より $(x_0,y_0) \in f$ だが $x_0 \notin \mbox{dom}f$ に矛盾する。よって $\mbox{dom}f=P_1$ より2番目が成り立つ。また $\mbox{dom}f \neq P_1$ のときも同様に3番目が成り立つことがわかる。 \\
これらの条件は互いに排反なので題意が示された。 \qed 

\section{順序数の定義}
\define 集合 $T$ が\textbf{推移的}とは、任意の元が $T$ の部分集合となることである。 ( $y \in x \in T \Rightarrow y \in T$ と言い換えられる)
\define \textbf{順序数}とは、 関係 $\in$ について整列集合かつ推移的な集合のことである(正則性公理から整列性は全順序性に仮定を弱められる)。 \\
順序数の二項関係 $<$ を $\alpha < \beta \iff \alpha \in \beta$ で定義する。

\lem\label{ordproperty}
\begin{enumerate}
    \item $0=\emptyset$ は順序数。
    \item $\alpha$ を順序数として $\beta \in \alpha$ ならば $\beta$ は順序数。
    \item $\alpha \neq \beta$ が順序数かつ $\alpha \subset \beta$ ならば $\alpha \in \beta$ 。
    \item $\alpha,\beta$ が順序数ならば $\alpha \subset \beta$ または $\beta \subset \alpha$ 。   
\end{enumerate}
\prf (1),(2) 明らか。 \\
(3) $\gamma$ を $\beta \setminus \alpha$ の最小元((2) よりこれは順序数)として $\alpha=\gamma$ を示す。まず $x \in \gamma$ とすると $\gamma$ の推移性から $x \in \beta$ 。ここで $x \notin \alpha$ ならば $x \in \beta \setminus \alpha$ と $\gamma$ の最小性から $x=\gamma$ または $\gamma \in x$ 。これはいずれも $x \in \gamma$ に反する。\\
次に $x \in \alpha$ とすると、仮定より $x \in \beta$ 。もし $x \notin \gamma$ ならば $x=\gamma$ または $\gamma \in x$ だが、$\alpha$ の推移性から $\gamma \in \alpha$ が得られ $\gamma$ の取り方に矛盾する。以上より $\alpha=\gamma$ 。
\\
(4) $\alpha \cap \beta=\gamma$ は明らかに順序数である。$\gamma$ が $\alpha,\beta$ のどちらでもないとすると、(3) より $\gamma \in \alpha,\gamma \in \beta$ なので $\gamma \in \gamma$ 。これは $\alpha$ が $\in$ での全順序集合であることに反する。  \qed
\\

補題 \ref{ordproperty} より次のことが確認できる(Exercise)
\begin{enumerate}
    \item 順序数全体の集合 $\mbox{On}$ は関係 $<$ で全順序となる。
    \item 順序数 $\alpha$ について $\alpha=\{\beta \colon \beta<\alpha\}$。
    \item 順序数の族 $C$ について $\cap C$ は順序数であり $\cap C \in C,\cap C=\inf C$ 。
    \item 順序数の集合 $X$ について $\cup X$ は順序数であり、 $\cup X=\sup X$ 。
    \item 順序数 $\alpha$ について $\alpha \cup \{\alpha\}$ も順序数であり $\alpha \cup \{\alpha\}=\inf\{\beta \colon \alpha<\beta\}$。
\end{enumerate}

\thm 任意の整列集合について、順序同型な順序数が一意に存在する。
\prf 補題 \ref{segment} より一意性はすぐに従うので、存在性を示す。$P$ を整列集合とし、$x \in P$ について $F(x)=\{F(y) \colon y \in P,y<x\}$ と定義し、像を $\alpha$ とする。\\
まず $\alpha$ の推移性をみる。 $\beta \in \alpha,\gamma \in \beta$ と仮定すると、定義から $\exists x \in P,\beta=F(x)$ であり $\exists y<x,\gamma=F(y)$ 。つまり $\gamma \in \alpha$ となるので良い。\\
次に整列集合であることをみる。$\in$ を関係とする全順序集合であることは $F$ の定義からすぐに分かる。 $S \subset \alpha$ を空でない部分集合とすると $\emptyset \neq \exists Y \subset P,S=F(Y)$ 。$P$ は整列集合だったので $Y$ の最小元が存在し、$F$ に写した値が $S$ の最小元となる。\\
最後に $F$ が順序同型であることを示せばよい。$x<y \Rightarrow F(x) \in F(y)$ より $F$ は順序を保つ単射であり、全射性は明らか。\qed \\

\define $\alpha+1=\alpha\cup \{\alpha\}$ を $\alpha$ の\textbf{後者} と呼ぶ。\\
$\alpha=\beta+1$ と書けるとき $\alpha$ を\textbf{後続順序数}、そうでないとき\textbf{極限順序数}と呼ぶ。$\emptyset$ を極限順序数とするかどうかは流儀がある。

\lem $\alpha$ が極限順序数であることと $\alpha=\sup_{\beta<\alpha}\beta$ は同値。
\prf $\gamma=\sup_{\beta<\alpha}\beta$ とおく。 $\gamma \leq \alpha$ は明らかなことに注意する。\\
$\alpha$ が極限順序数のとき $\alpha \leq \gamma$ を示す。$\gamma<\alpha$ とすると、 $\alpha$ が極限順序数であることから $\gamma<\gamma+1<\alpha$ ( $\beta<\alpha \Rightarrow \beta+1 \leq \alpha$ に注意)。よって $\gamma<\gamma$ だがこれは全順序性に矛盾。\\
逆に $\alpha=\gamma$ とする。$\forall \beta< \alpha=\gamma$ について、$\gamma$ の定義から $\beta<\exists \delta<\alpha$ 。よって $\beta<\beta+1 \leq \delta<\alpha$ より $\alpha$ は $\beta\cup \{\beta\}$ と書けない。 \qed \\

\define 無限公理から $\emptyset,\emptyset\cup\{\emptyset\},\emptyset\cup\{\emptyset\}\cup\{\emptyset\cup\{\emptyset\}\},\ldots$ を元とする集合の存在が保証され、順序数の公理を満たす。これを $\omega$ と書く。 \\

\section{順序数の算法}

\define $\alpha>0$ を極限順序数、$\{a_\xi\}_{\xi<\alpha}$ を非減少な順序数の列としたとき、その極限を $\lim_{\xi \to \alpha} a_{\xi}=\sup\{a_\xi \colon \xi<\alpha\}$ で定める。

\define $\alpha,\beta$ の和を以下で帰納的に定める。
\begin{enumerate}
    \item $\alpha+0=\alpha$
    \item $\alpha+(\beta+1)=(\alpha+\beta)+1$
    \item $\alpha+\beta=\lim_{\xi \to \beta} \alpha+\xi$ \ ($\beta>0\colon$ limit)
\end{enumerate}
$\alpha \cdot \beta,\alpha^\beta$ についても同様にいい感じで定義する。 \\

分配法則などの各種性質は超限帰納法で証明できる。
\thm(超限帰納法) 順序数を引数にもつ論理式 $\phi$ について
\begin{enumerate}
    \item $\phi(0)$:True
    \item $\phi(\alpha) \Rightarrow \phi(\alpha+1)$
    \item $\forall \beta<\alpha,\phi(\beta) \Rightarrow \phi(\alpha)$\ ($\alpha\colon$ limit)
\end{enumerate}
が成り立つなら、任意の順序数 $\alpha$ について $\phi(\alpha)$ はTrue。
\prf  $\phi(\alpha)$ がFalseとなる最小の $\alpha$ をとって上の条件を適用する。 \qed

\thm\label{NormalForm}(Cantor標準形) $\alpha>0$ は $n \geq 1,\alpha \geq \beta_1>\cdots>\beta_n,k_1,\cdots,k_n \in \omega$ を用いて $\alpha=\omega^{\beta_1} \cdot k_1+\cdots +\omega^{\beta_n}\cdot k_n$ と一意的に表される。
\prf $\alpha$ の帰納法による。 $\alpha=1$ のときは $1=\omega^0\cdot 1$ より良い。\\
$\alpha>1$ について $\beta<\gamma \Rightarrow \omega^\beta<\omega^\gamma$ を用いると $\alpha \leq \omega^\alpha<\omega^{\alpha+1}$ 。よって $\alpha<\omega^\xi$ なる $\xi$ が存在するので最小元をとってくる。$\xi$ が極限順序数だとすると $\sup$ の定義からより小さな元を取れるので $\xi$ は後続型。 \\
$\xi=\beta_1+1$ とおくと $\omega^{\beta_1}\leq \alpha<\omega^{\beta_1}\cdot \omega$ 。よって $\alpha<\omega^{\beta_1}\cdot \eta$ なる最小の $\eta$ は2以上の自然数なので $\eta=k_1+1$ と書ける。 \\
このとき $\alpha=\omega^{\beta_1}\cdot k_1+\alpha_1(0 \leq \alpha_1<\omega^{\beta_1}<\alpha)$ となるので $\alpha_1$ に帰納法の仮定を適用すれば存在性がわかる。\\
一意性も同様に $\alpha$ の帰納法を用いればよい。 \qed

\chapter{基数}






\end{document}